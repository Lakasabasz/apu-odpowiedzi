\documentclass[11pt]{article}

\usepackage{amsmath}
\usepackage{babel}
\usepackage[a4paper, margin={0.75in}]{geometry}
\usepackage{minted}

\begin{document}
    \begin{center}
        \LARGE{Odpowiedzi na pytania do egzaminu z APU} \\
        \Large{Wykłady 1--3}
    \end{center}

    \vspace{2cm}
    
    \section{Wykład}
    \begin{enumerate}
        \item Definicja procesów uczenia \\
        Nie ma jednej definicji procesów uczenia się:
        \begin{itemize}
            \item ``Uczenie się oznacza zmiany w systemie, które mają charakter adaptacyjny
            w tym sensie, że pozwalają systemowi wykonać za następnym razem takie
            same zadanie lub zadania podobne bardziej efektywnie'' - Herbert Simon
            (1983)
            \item ``System uczący się wykorzystuje zewnętrzne dane empiryczne w celu tworzenia
            i aktualizacji podstaw dla udoskonalania działania na podobnych danych w przyszłości
            oraz wyrażania tych podstaw w zrozumiałej i symbolicznej postaci'' - Donald Miche
            (1991)
            \item ``Uczenie sie to konstruowanie i zmiana reprezentacji doświadczanych
            faktów.
            W ocenie konstruowanych reprezentacji bierze się pod uwagę:
            \begin{enumerate}
                \item wiarygodność - określa stopień w jakim reprezentacja odpowiada rzeczywistości;
                \item efektywność - charakteryzuje przydatność reprezentacji do osiągania danego celu;
                \item poziom abstrakcji - odpowiada zakresowi szczegółowości i precyzji pojęć używanych
                w reprezentacji; określa on tzw.\ moc opisową reprezentacji.
                Reprezentacja jest rozumiana jako np.\ opisy symboliczne, algorytmy,
                modele symulacyjne, plany obrazy.'' - Ryszard Michalski (1986)
            \end{enumerate}
            \item Elementem wspólnym tych definicji są: Wejście (dane empiryczne), miara oceny (Zmiany 
            i poprawa działania) oraz postulat zdobywania wiedzy, reprezentowania jej wewnątrz systemu
            i stosowania jej do wykonania zadania (nacisk na zrozumiałość reprezentacji)
        \end{itemize}
        \item Przykłady problemów rozwiązywanych przez systemy uczące się \\
        \begin{itemize}
            \item Uczenie się rozpoznawania mowy
            \item Uczenie się kierowania pojazdem (np.\ ALVINN)
            \item Uczenie się klasyfikacji obiektów astronomicznych (NASA Sky Survey)
            \item Uczenie się rozgrywania pewnych gier
            \item Uczenie się rozpoznawania chorób na podstawie symptomów
            \item Uczenie się rozpoznawanie pisma na podstawie przykładów
            \item Uczenie się klasyfikowania tekstów do grup tematycznych
            \item Uczenie się aproksymacji nieznanej funkcji na podstawie próbek
            \item Uczenie się odnajdowania drogi w nieznanym środowisku
            \item Automatyczne odkrywanie zależności funkcyjnych w danych
            \item Przewidywanie trendów w danych finansowych
        \end{itemize}
        \clearpage
        \item Motywacje dla budowy systemów uczących się
        \begin{itemize}
            \item Zadania eksploracji i analizy danych, gdzie duże rozmiary zbiorów
            danych uniemożliwiają ich analizę w sposób nieautomatyczny (np.
            ekonomiczne lub medyczne bazy danych)
            \item Środowiska gdzie system musi się dynamicznie dostosowywać do
            zmieniających się warunków (np.\ systemy sterowania)
            \item Problemy które są złożone, trudne do opisu i często nie posiadają
            wystarczających modeli teoretycznych albo ich uzyskanie jest bardzo
            kosztowne lub mało wiarygodne.
        \end{itemize}
        \item Klasyfikacja metod maszynowego uczenia się
        \begin{itemize}
            \item Uczenie indukcyjne - na podstawie znanych faktów i obserwacji tworzona
            jest uogólnienie, które próbuje dopasować wszystkie znane fakty do hipotezy
            \item ``Nabywanie umiejętności'' - optymalizacja zastosowania już posiadanej
            wiedzy w sposób zwiększający jej efektywność
        \end{itemize}
        \item Tworzenie modelu uczenia maszynowego
        \begin{enumerate}
            \item Przygotowanie danych (zebranie danych, uzupełnienie braków, normalizacja)
            \item Zdefiniowanie zadania (regresja, klasyfikacja, grupowanie, inne)
            \item Wybór metody (regresja liniowa, logistyczna, drzewa decyzyjne,
            sieci neuronowe)
            \item Strojenie parametrów
            \item Ocena modelu
        \end{enumerate}
        \item Język R. Wykonywanie instrukcji\\
        Sekwencyjne, możliwe wpisywanie kolejnych poleceń z konsoli lub z pliku skryptu
        o rozszerzeniu *.R.
        W konsoli występuje znak zachęty \mintinline{text}|>|.
        Komentarz tworzony jest przez \mintinline{r}|#|.
        Zapisywanie ma postać \mintinline{r}|zmienna <- wartość|.
        \item Korzystanie z pomocy R
        \begin{itemize}
        	\item \mintinline{r}|help(max)| - klasyczna pomoc
        	\item \mintinline{r}|?max| - skrócona wersja
        	\item \mintinline{r}|example(max)| - przykłady użycia
        	\item \mintinline{r}|RSiteSearch("max function")| - przeszukiwanie forum
        	\item \mintinline{r}|apropos("max", mode = "function")| - wyszukiwanie z funkcji z nazwą "max"
        	\item \mintinline{r}|data()| - nazwy obiektów w pakiecie
        	\item \mintinline{r}|vignette()| - pdf z dokumentacją
        \end{itemize}
        \item Zarządzanie obszarem roboczym R
        \item Pakiety rozszerzające R
        \item Skalary i wektory R
        \item Ramka danych R
        \item Przegląd wykresów
        \item Język R i uczenie maszynowe
    \end{enumerate}
    
    \section{Wykład}
    \begin{enumerate}
        \item Analiza eksploracyjna i analiza potwierdzająca
        \item Czym są dane w uczeniu maszynowym?
        \item Wnioskowanie o typach danych w kolumnach
        \item Podsumowania liczbowe w R
        \item Średnie, mediany i dominanty w R
        \item Kwantyle w R
        \item Odchylenia standardowe i wariancje w R
        \item Eksploracyjne wizualizacje danych
        \item Wizualizowanie powiązań pomiędzy kolumnami
        \item Klasyfikacja; Zdefiniowanie zadania
        \item Trening i testowanie klasyfikacji
        \item Kryteria porównawcze metod klasyfikacji
        \item Metody klasyfikacji
        \item Drzewa decyzyjne
        \item Funkcje testu w celu konstruowania drzew decyzyjnych
        \item Konstrukcja drzew decyzyjnych
        \item Problem brakujących wartości przy konstruowaniu drzew decyzyjnych
        \item Analiza ROC jakości klasyfikacji
        \item Krzywe ROC
        \item Czułość, a specyficzność klasyfikacji binarnej
        \item Konstruowanie krzywych ROC
        \item Pakiet ROCR
    \end{enumerate}
    
    \section{Wykład}
    \begin{enumerate}
        \item Wieloatrybutowe problemy decyzyjne
        \item Proces analitycznej hierarchizacji problemu decyzyjnego
        \item Kroki rozwiązywania problemu AHP
        \item Podstawy wieloatrybutowej teorii użyteczności
        \item Agregacja ocen z wykorzystaniem macierzy porównań parami
        \item Skala preferencji względnej
        \item Ocena spójności macierzy porównań parami
        \item Krok V – obliczenie priorytetów AHP
        \item Obliczanie przybliżonego wektora własnego macierzy porównań parami
        \item Inne metody rozwiązywania problemu AHP
        \item Przykłady zastosowań metody AHP
    \end{enumerate}
\end{document}