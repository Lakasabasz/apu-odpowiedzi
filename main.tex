\documentclass[11pt]{article}

\usepackage{amsmath}
\usepackage{babel}
\usepackage[a4paper, margin={0.75in}]{geometry}


\begin{document}
    \begin{center}
        \LARGE{Odpowiedzi na pytania do egzaminu z APU} \\
        \Large{Wykłady 1--3}
    \end{center}

    \vspace{2cm}
    
    \section{Wykład}
    \begin{enumerate}
        \item Definicja procesów uczenia
        \item Przykłady problemów rozwiązywanych przez systemy uczące się
        \item Motywacje dla budowy systemów uczących się
        \item Klasyfikacja metod maszynowego uczenia się
        \item Tworzenie modelu uczenia maszynowego
        \item Język R. Wykonywanie instrukcji
        \item Korzystanie z pomocy R
        \item Zarządzanie obszarem roboczym R
        \item Pakiety rozszerzające R
        \item Skalary i wektory R
        \item Ramka danych R
        \item Przegląd wykresów
        \item Język R i uczenie maszynowe
    \end{enumerate}
    
    \section{Wykład}
    \begin{enumerate}
        \item Analiza eksploracyjna i analiza potwierdzająca
        \item Czym są dane w uczeniu maszynowym?
        \item Wnioskowanie o typach danych w kolumnach
        \item Podsumowania liczbowe w R
        \item Średnie, mediany i dominanty w R
        \item Kwantyle w R
        \item Odchylenia standardowe i wariancje w R
        \item Eksploracyjne wizualizacje danych
        \item Wizualizowanie powiązań pomiędzy kolumnami
        \item Klasyfikacja; Zdefiniowanie zadania
        \item Trening i testowanie klasyfikacji
        \item Kryteria porównawcze metod klasyfikacji
        \item Metody klasyfikacji
        \item Drzewa decyzyjne
        \item Funkcje testu w celu konstruowania drzew decyzyjnych
        \item Konstrukcja drzew decyzyjnych
        \item Problem brakujących wartości przy konstruowaniu drzew decyzyjnych
        \item Analiza ROC jakości klasyfikacji
        \item Krzywe ROC
        \item Czułość, a specyficzność klasyfikacji binarnej
        \item Konstruowanie krzywych ROC
        \item Pakiet ROCR
    \end{enumerate}
    
    \section{Wykład}
    \begin{enumerate}
        \item Wieloatrybutowe problemy decyzyjne
        \item Proces analitycznej hierarchizacji problemu decyzyjnego
        \item Kroki rozwiązywania problemu AHP
        \item Podstawy wieloatrybutowej teorii użyteczności
        \item Agregacja ocen z wykorzystaniem macierzy porównań parami
        \item Skala preferencji względnej
        \item Ocena spójności macierzy porównań parami
        \item Krok V – obliczenie priorytetów AHP
        \item Obliczanie przybliżonego wektora własnego macierzy porównań parami
        \item Inne metody rozwiązywania problemu AHP
        \item Przykłady zastosowań metody AHP
    \end{enumerate}
\end{document}